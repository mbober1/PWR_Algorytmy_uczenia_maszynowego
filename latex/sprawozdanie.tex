\documentclass[12pt,a4paper]{article}
\topmargin -1.6cm
\addtolength{\textheight}{4cm}
\textwidth  15.5cm

\leftmargin      5mm
\rightmargin     5mm
\oddsidemargin   5mm
\evensidemargin  5mm

\usepackage{hyperref}
\usepackage{polski}
\usepackage[utf8]{inputenc}
\usepackage{graphicx}
\usepackage{units}
\usepackage{sty/style}
\usepackage{float}
\usepackage{mathtools}



\projekt{Projekt}
\autor{Marcin Bober, 249426}
\przedmiot{Algorytmy Uczenia Maszynowego}
\prowadzacy{Mgr inż. Marcin Ochman}

\begin{document}
\pdfpageheight   297mm
\pdfpagewidth    210mm

\StronaTytulowa
\SpisTresci

\pagebreak

\section{Opis problemu}
  \subsection{Temat}
  Projekt zakłada wykorzystanie uczenia maszynowego w celu rozpoznawania gatunku wina ze względu na jego analize chemiczną składu.

  \subsection{Algorytmy}
  W celu zrealizowania powyższego celu, zostaną użyte 3 różne algorytmy uczenia maszynowego oraz ich synteza.

  \begin{itemize}
    \item algorytm SVC,
    \item algorytm One-Class SVM,
    \item algorytm MLPClassifier.
  \end{itemize}

  \subsection{Terminarz}
  Projekt podzielony jest na sześć cześci i każda z nich wykonywana będzie w ściśle określonym przedziale czasu zamieszczonym w tabeli \ref{terminarz}

  \begin{table}[h!]
    \centering
    \begin{tabular}{ r | l }
    28.03 & Przygotowainie środowiska i repozytorium, poszerzanie wiedzy. \\
    11.04 & Praca nad algorytmem 1 \\
    09.05 & Praca nad algorytmem 2 \\
    23.05 & Praca nad algorytmem 3 \\
    06.06 & Praca syntezą algorytmów \\
    20.06 & Prezentacja wyników \\
    \end{tabular}
    \label{terminarz}
    \caption{Terminy zadań}
  \end{table}

\end{document}